\documentclass{article}
\renewcommand\familydefault{\sfdefault}
\usepackage[margin=1in]{geometry}
\setlength{\parindent}{0em}
\setlength{\parskip}{1ex}
\usepackage{xcolor}
\renewcommand{\thesection}{Unit \arabic{section}}
\usepackage{enumitem}
\renewcommand{\labelitemi}{$\vcenter{\hbox{\tiny$\bullet$}}$}
%\setenumerate{ label={\arabic{section}.\arabic*}, topsep=0ex, itemsep=0ex, }
%\setlist[enumerate,2]{ label={\arabic{section}.\arabic{enumi}.\arabic*}, itemsep=0ex, topsep=0ex, }


\begin{document}

\section{Energy Generation}
\begin{enumerate}
    \item Basics
        \begin{enumerate}
            \item Energy equilibrium
            \item Nuclear interactions
            \item Nuclear reaction rates
            \item Energy release in nuclear reactions
            \item Binding energy
        \end{enumerate}
    \item Hydrogen burning
        \begin{enumerate}
            \item PP-I chain
            \item PP-II and PP-III chain
            \item CNO cycle
        \end{enumerate}
    \item Things not discussed
\end{enumerate}

\newpage
\section{Hydrostatics}
\begin{enumerate}
    \item Are stars approximately a one-fluid plasma?
    \item Time scales of stars
        \begin{enumerate}
            \item Dynamical timescale
            \item Thermal timescale
            \item Nuclear timescale
        \end{enumerate}
    \item Equation of state
        \begin{enumerate}
            \item Preliminaries
            \item Mean molecular weight (${\mu}$)
            \item Ideal monatomic gas
            \item Completely degenerate gas
            \item Partially degenerate gas
            \item ---
            \item Radiation Pressure
            \item Density-temperature equation of state landscape
            \item Thermodynamics of an ideal gas
            \item Mixture of ideal gas and radiation: pressure effects
            \item Mixture of ideal gas and radiation: ionization effects
        \end{enumerate}
    \item Hydrostatic equilibrium
        \begin{enumerate}
            \item Derivation
        \end{enumerate}
    \item The Virial Theorem
    \item Polytropes
        \begin{enumerate}
            \item Motivation and derivation
            \item Lane-Emden equation
            \item Polytrope solutions
        \end{enumerate}
\end{enumerate}

\newpage
\section{Energy Transport}
\begin{enumerate}
    \item Radiation
        \begin{enumerate}
            \item Basics
                \begin{itemize}
                    \item Efficiency of energy transfer depends on:
                        \begin{enumerate}[label=\arabic{*}.]
                            \item Temperature gradient of the solar interior
                            \item Mean free path for photons: $\ell = \frac{1}{\kappa\rho}$
                        \end{enumerate}
                \end{itemize}
            \item Diffusion
                \begin{itemize}
                    \item Fick's Law: $ F = -D \nabla_{r} n $;
                        $ D = \frac{1}{3}\overline{v}\ell $
                        = diffusion coefficient
                    \item Temperature gradient: $ \frac{\mathrm{d}T}{\mathrm{d}r} = -\frac{3}{16\pi ac} \frac{\kappa\rho}{r^{2}}\frac{L}{T^{3}} $
                \end{itemize}
            \item Frequency dependence of radiation
                \begin{itemize}
                    \item Rosseland mean opacity, $\kappa_{R}$
                \end{itemize}
            \item Opacity sources
                \begin{itemize}
                    \item $\kappa = \kappa_{0} \rho^{n} T^{-s}$ [cm$^{2}$ g$^{-1}$]
                    \item \textit{Kramer's opacities} scale as $ \kappa \sim \rho T^{-3.5} $
                        ($n = 1, s = 3.5$)
                    \item Four main sources:
                        \begin{enumerate}
                            \item Compton/Thomson scattering (photon-electron)
                            \item Free-free absorption (Kramer's, core)
                            \item Bound-free absorption (Kramer's, surface)
                                \begin{itemize}
                                    \item H$^{-}$: Sensitive to metallicity,
                                        needs free electrons,
                                        $\nu$ = IR and up
                                \end{itemize}
                            \item Bound-bound absorption
                        \end{enumerate}
                \end{itemize}
            \item Consequences (See figures)
            \item Eddington Luminosity
                \begin{itemize}
                    \item Maximum luminosity a star can have and still balance gravity
                \end{itemize}
            \item Final tools:
                \begin{itemize}
                    \item $\nabla \equiv \frac{d ln T}{d ln P}$ = True driving gradient:
                    \item $\nabla_{rad}$ = Radiation gradient, slope required if all
                        luminosity was carried by radiation through diffusion
                    \item $\nabla_{ad}$ = Adiabatic gradient, rate that temperature of
                        a parcel of gas changes with height
                \end{itemize}
        \end{enumerate}
    \item Conduction:
        $ \frac{1}{\kappa_{tot}} = \frac{1}{\kappa_{R}} + \frac{1}{\kappa_{cond}}$
    \item Convection
        \begin{enumerate}
            \item The convective instability:
                Schwarzschild criterion, Brunt-Vaisala frequency
            \item Another useful formulation:
                \[
                    T dS = dU + PdV
                    \]
                \[
                    \frac{dS}{dr} = c_{P} (\nabla - \nabla_{ad})
                    \frac{d ln P}{dr}
                    \]
                Entropy is constant for adiabatic processes
            \item Semiconvection
                \[
                    N^{2} = \frac{g^{2}\rho}{P}
                    \left( \nabla_{ad} - \nabla + \nabla_{\mu} \right)
                    \]
                \textit{Ledoux criterion}:
                Schwarzschild criterion is satisfied, but the medium is still
                stable due to a positive composition gradient
            \item One more useful formulation
            \item Physical conditions for convection onset
                \begin{itemize}
                    \item L/m is large
                    \item $\kappa$ is large
                    \item $\rho/T^{3}$ is large
                    \item $\nabla_{ad} = 1 - 1/\gamma $ is small
                \end{itemize}
            \item Mixing length theory
            \item Convective overshoot: momentum carries convection beyond the layer
                where parcels become stable, possibly depositing mixed material in
                stable regions.
            \item Depth of outer convection zones
                \begin{itemize}
                  \item $T_{eff}$
                  \item chemical abundances
                \end{itemize}
                Higher metallicity $\longrightarrow$ higher opacity and deeper
                convection zones
        \end{enumerate}
\end{enumerate}

\newpage
\section{The Main Sequence}
\begin{enumerate}
    \item Summary of stellar structure
    \item Homology relations for stars in \textbf{radiative equilibrium}
        \begin{enumerate}
            \item Basic idea
            \item Dependence on mass
            \item Dependence on ${T_{eff}}$
            \item Dependence on mean molecular weight (${\mu}$)
            \item Dependence on heavy metal abundances
            \item Contracting stars in radiative equilibrium
            \item Convective stars
        \end{enumerate}
    \item Evolution on the main sequence
        \begin{enumerate}
            \item Low-mass stars
            \item High-mass stars
            \item A note about \emph{very} low mass stars
        \end{enumerate}
    \item Summary of Main-Sequence properties
\end{enumerate}

\newpage
\section{The Post Main Sequence}
\begin{enumerate}
    \item General considerations
        \begin{enumerate}
            \item Sch\.{o}nberg-Chandrasekhar Limit
            \item The subgiant branch
        \end{enumerate}
    \item Toward and up the RGB
        \begin{enumerate}
            \item High-mass stars
            \item Low-mass stars
            \item RGB properties
            \item Summary
        \end{enumerate}
    \item Helium burning
        \begin{enumerate}
            \item Quick tour of non-hydrogen nuclear reactions
            \item Horizontal Branch
            \item Location of the ZAHB
            \item Horizontal branch evolution
            \item Asymtotic giant branch
            \item Thermal pulses
        \end{enumerate}
    \item Last stages of evolution: low-mass stars
        \begin{enumerate}
            \item Production of $s$ elements
            \item Planetary nebula
            \item White dwarfs
            \item Further WD properties
            \item Type Ia supernovae
        \end{enumerate}
    \item Last stages of evolution: high-mass stars
        \begin{enumerate}
            \item Nuclear burning
            \item Type II supernova -- core collapse
            \item Neutron star
            \item Black hole
        \end{enumerate}
\end{enumerate}
\end{document}
